\documentclass[a4paper,10pt]{report}
\begin{document}
\title{DEMYSTIFYING THE USE OF FREE TO AIR TECHNOLOGY FOR TELEVISON CHANNEL ACCESS}
\author{Akora Brian Marco}
\date{\today}
\maketitle
\section{Abstract}
\paragraph{\textrm{This report attempts to explain how an ordinary citizen, especially those in upcountry locations can obtain access to Free to Air technologies in order to avail to themselves  a vast array of television and in addition, radio services without having to opt for a Pay-Tv subscription option.}}
\section{Introduction}
\paragraph{Until as recently as June 2015, Uganda had space for both analog and digital mode of access to television stations. The analog mode involved the tuning of stand-alone television sets, aided by a simple two pronged antenna that normally sat on top of the television set; with the television channel search and storage conducted wholly by the television set. All one had to do was move through a television’s menu and access the auto search. The TV would then search for available stations whose frequencies it could access and that was it.}
\paragraph{The Uganda Communications Commission then instated a directive that would see the entire country effect an analog-digital migration. Digital television merely involves the integration of a set top box into also referred to as a decoder with ones television in order to receive television services. The digital aspect is an upgrade on the more traditional Analog setup. It features more television stations, better picture quality, and better sound quality. One’s mind has probably shifted to major players in this market such as DSTV, GoTv, Startimes, among others. Case in point here is; al these are Pay-Tv service providers in nature. This means that an interested party would have to pay subscription fees to receive Tv services. }
\section{How to access free to air digital TV services}
\paragraph{There are available options for citizens that would like to access digital television without being subjected to weekly or monthly TV bills. The technology I would like to describe, however, is more suited to persons in upcountry locations that not only seek access to local stations but also a vast array of international TV channels. It is categorized under the Free to Air technology.}
\paragraph{ For a start, one needs a sizeable satellite dish and a free to air decoder. I can point out to two established providers of Free to Air decoder technology, namely Alfa Gold Digital and Astrovox, with the former able to have up to 4200 television and radio channels while the latter racks up to 5000 television and radio channels.
The satellite dish is usually large and is sold as a disintegrated entity requiring assembly upon set-up. It is preferable to assemble it outdoors with the help of a technician. On completion of assembly, a platform may be built for the dish, advisably in open space, devoid of obstacles such as high rise buildings, trees etcetera and with a full view of the sky. On the hand it may be set upon the roof, or be left on the ground.}
\paragraph{It is advisable to complete the installation with all the associated components of the installation (TV, decoder) close to the satellite dish by making use of long wired extension cables or external wall sockets if available. The decoder is then connected to the LNB which usually hangs at the center of the dish via the long cable that is normally provided with the rest of the hardware. The decoder is then connected to the television and powered on.}
\paragraph{The decoder initialization process normally comes along with colorful interfaces but straight away, the message banner “No TV services” is displayed. With the help of a technician, one is required to immediately access the menu and then, settings. Next, any option that is installation related should be chosen. This will give the user access to an interface that displays a list of satellites for example Hotbird, Eutelsat, Intelsat, Arabsat among others, technical options like Transponder number, Symbol Rate, Frequency.}
\paragraph{The technician is usually required to manually input the corresponding values for symbol rates, frequency etc., which are usually pre-recorded, into this interface before any channel search is done.  The position of the dish is then altered by rotating it about a fixed position while modifying its inclination with the help of the adjustable inclination modifier, until the signal strength, signal quality meters record reasonably high percentages. It is advisable to alter the position of the dish until much ‘healthier’ percentages are recorded after which the position of the dish, both on the ground/platform and the inclination modifier are marked. This is usually the most exasperating part of the installation! }
\paragraph{With considerably high percentages for signal quality and strength recorded, a channel search is commenced. The decoder pulls in as many stations as are on that symbol rate and frequency with the TV channels of quality as good as those offered by Pay-Tv service providers in terms of picture and sound quality.}
\section{Conclusion}
\paragraph{Free to Air services give users access to European, American, Arabic Television stations, in addition to local content. The only outstanding challenge with this arrangement is, the positions of the satellites are subject to alteration; therefore one must always keep an updated list of Satellites, Frequencies and Symbol Rates in order to keep enjoying the free to air television and radio stations. Otherwise, it is a technology that is worth ones while!}
\end{document}